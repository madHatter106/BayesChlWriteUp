\documentclass[10pt]{article}
\usepackage[utf8]{inputenc}
\usepackage{amsmath}
\usepackage{amsfonts}
\usepackage{tikz}
\usepackage{amssymb}
%\usepackage{fullpage}
\usepackage[top=1in, bottom=1in, left=1in, right=1in]{geometry}
\usepackage{graphicx}
\usepackage{hyperref}
\usepackage{subcaption}
\usepackage{float}
\usepackage[section]{placeins}
\graphicspath{{./FigJar/}}
\begin{document}
\author{Erdem M. Karaköylü}
\title{A Bayesian Approach to OC4}
\date{\today}
\maketitle
\tableofcontents
\newpage

\newcommand{\reddash}{\raisebox{2pt}{\tikz{\draw[-,red,dashed,line width=1.2pt](0,0) -- (5mm,0);}}}
\newcommand{\blkdash}{\raisebox{2pt}{\tikz{\draw[-,black,dashed,line width=1.2pt](0,0) -- (5mm,0);}}}
\newcommand{\blksold}{\raisebox{2pt}{\tikz{\draw[-,black,solid,line width=1.2pt](0,0) -- (5mm,0);}}}

\section{Introduction}
	\subsection{Background}
		\begin{itemize}
		\item Necessity for estimating chlorophyll
		\item State of current chlorophyll algorithms
		\item Basic empirical form
		\begin{align}
		log_{10}\left(chlor_a\right) = a_0 + \sum_{i=1}^ja_ilog_{10}\left(\frac{max\left(Rrs	\left(\lambda_{blue}\right)\right)}{Rrs\left(\lambda_{green}\right)}\right)
		\end{align}
		\item Problems with current algorithms:
		\begin{itemize}
			\item collinearity of inputs
			\item poor performance in coastal
			\item maximum likelihood estimation approach \rightarrow increased odds of overfitting in particular in the current state of scant field data compared to remote sensing data
		\end{itemize}
	\end{itemize}
	\subsection{Proposed framework}
		\subsubsection{Basis reduction via PCA}
		\begin{itemize}
			\item PCA of Rrs to reduce overlap of information between predictor variables
		\end{itemize}
		\subsubsection{Bayesian framework for chlorophyll estimation from remote sensing data}
			\begin{itemize}
				\item transparent construction of models with explicit formulation of assumptions,
				\item assumptions/background information codified as priors,
				\item feasibility of priors verifiable before data collection via prior predictive checks
				\item built-in structure for selecting relevant features,
				\item posterior distribution as rich information structure from which to estimate parameter uncertainty as well as output prediction uncertainty,
				\item predictive ability of model assessed via posterior predictive checks
				\item multiple models encouraged by bayesian workflow,
				\item evaluation/comparison between models using both information about model complexity and posterior distribution (WAIC),
			\end{itemize}
		\subsubsection{Reproducibility}
			\begin{itemize}
				\item iterative process of bayesian framework relies on reproducibility for progress
				\item code available via github
				\item data available via osf
			\end{itemize}

\section{Methods}
	\subsection{Model Development}		
	\

\end{document}
			
		%	\begin{figure}[H]
		%		\centering
		%		\includegraphics[scale=0.5]{randomMapPts.png}
		%		\caption{Random sampling of 18 geographic locations}
		%	\end{figure}
		